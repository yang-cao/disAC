\vspace{2ex}
%%%%%%%% Section 3 %%%%%%%%
\section{Ranking Cache Schemas}
\label{sec-rank}


We next develop criteria for ranking \bdss. We then
formulate the \bds selection problem  and
prove its intractability.


\warn{\stitle{Scan-free evaluability}.
One naturally wants to find \bdss that can make as many queries
scan-free as possible to benefit from scan-free evaluation of
\baav databases.  However, this alone does not make good \bdss.
For instance, recall query $Q_{1}$ and \bss $\kb{\at{T}}_{1}$
and $\kb{\at{T}}_{1}'$ from Example~\ref{exa-baav}; $Q_{1}$ is
scan-free over both $\kb{\at{T}}_{1}$ and $\kb{\at{T}}_{1}'$
while its scan-free plan over $\kb{\at{T}}_{1}$ is superior than
that over $\kb{\at{T}}_{2}$ since it fetches fewer attributes
and hence access fewer values. Moreover, }






We propose four criteria to quantify
the quality of \bdss. 
Consider workload $\Q$ over database schema $\R$, where (a) each
parametric query $Q$ in $\Q$ is associated with an importance
weight $w_{Q}$ representing, \eg the frequency of $Q$ 
instantiated and executed, and (b) each relation $R$ in $\R$
has a coefficient $l_{R}$ denoting, \eg its update frequency.

\eetitle{(1) Scan-free evaluability}.
The first criterion, referred to as the {\em scan-free
evaluability} of $\kb{\R}$ for $\Q$ and denoted by $\usf(\Q,
\kb{\R})$, measures the number of queries in $\Q$ that are not
scan-free over $\kb{\R}$. More specifically, $\sf(\Q, \kb{\R})$
is defined as
\vspace{-0.7ex}
\[\usf(\Q, \kb{\R}) = \textstyle \sum_{Q\in \Q} w_{Q} * (1 - \sf(Q, \kb{\R})),
\vspace{-0.7ex}\]
where $\sf(Q,\ak \kb{\R})$ is 1 if query
$Q$ is scan-free over $\kb{\R}$
and is 0 otherwise. Intuitively, $\usf(\Q,\ak \kb{\R})$ measures
the total number of queries in $\Q$ that are not scan-free over
$\kb{\R}$,
and it is magnified by their instantiation frequencies.
By minimizing $\usf(\Q,\ak \kb{\R})$, we favor queries that are
frequently executed and aim to make as many of them
  scan-free as possible over $\kb{\R}$.

\eetitle{(2) Data access}.
The second criterion measures
the amount of data fetched over $\kb{\R}$.
%
To illustrate this,
recall $Q_{1}$ and the \bs $\kb{\at{T}}'$ from
Example~\ref{exa-baav},
and \bds $\kb{\R}_{1}$ from Example~\ref{exa-mapping}. Let $\kb{\R}'_{1}$
be a revision of $\kb{\R}_{1}$ with $\kb{\at{T}}_{1}$ replaced by
$\kb{\at{T}}'$. Then, as discussed in Example~\ref{exa-baav},
$Q_{1}$ is also scan-free over $\kb{\R}'_{1}$  but it accesses many
more data values than over $\kb{\R}_{1}$. 
This motivates us to further minimize the size of accessed data, 
which can be abstracted by the
degrees of the \baav instances of $\kb{\R}$.

\vspace{0.36ex}
More specifically,
for a $\get(k)$ operation over a \bs $\kb{R}$, the size
of its accessed data is determined by the degree of the \baav
instance of $\kb{R}$, which depends on the number of the
$Y$-attributes of $\kb{R}$.
For example, $\at{T}_{1}'$ of $\kb{\R}_{1}'$ has fewer
$Y$-attributes than $\at{T}_{1}$ of $\kb{\R}_{1}$ and hence
$Q_{1}$ access less data values over $\kb{\R}_{1}'$.
\looseness = -1

\vspace{0.36ex}
Therefore, we can characterize the data access of
$\kb{\R}$ by $\Q$ in terms of its $Y$-attributes
and minimize it, as follows:
\vspace{-0.7ex}
\[
\textstyle \daccess(\Q, \kb{\R}) = \textstyle
\sum_{\kb{R}\bschema{X}{Y} \in \kb{\R}} |Y|.
\vspace{-1ex}\]


\eetitle{(3) Space}.
The third criterion measures
the size of \baav databases of $\kb{\R}$, for which the need
is evident.
We assume that there exists an oracle $\esize(\kb{R})$
that estimates the size of the \baav instance of \bs $\kb{R}$ in $\kb{\R}$.
%for the database of interest.
Examples of $\esize(\kb{R})$ include
the total number of attributes in $\kb{R}$ (\ie measure of the
size at the schema level) or the conventional histogram-based
methods to estimate the number of values in the instance. Given
$\esize(\kb{R})$, we minimize $\size(\Q, \kb{\R})$ of
\baav database of $\kb{\R}$ defined as
\vspace{-0.7ex}
\[\size(\Q,\kb{\R}) = \textstyle \sum_{\kb{R}\in
  \kb{\R}}\esize(\kb{R}).
\vspace{-0.7ex}\]

By default, we assume that $\esize(\kb{R})$ is estimated by the
number of attributes in $\kb{R}$. All of our results hold
for more sophisticated size estimations as long as they
satisfy $\esize(\kb{R}) \leq \esize(\kb{R'})$ when attributes of
$\kb{R}$ are also attributes of $\kb{R'}$.

\eetitle{(4) Maintenance}.
The last one captures the maintenance cost of \baav databases.
We use the update frequency of \bss to measure its incremental
maintenance cost.
For \bs $\kb{R}$ that is mapped from a single relation schema
$R$, its update cost is measured simply as $l_{R}$, \ie the
update frequency of $R$; when $\kb{R}$ contains attributes mapped
from multiple relations, its update cost is measured as the
product of the update frequencies of the relevant relations.
Formally, this is defined as
$\update(\Q, \kb{\R})$:
\vspace{-0.7ex}
\[\update(\Q,\kb{\R}) = \textstyle\sum_{\kb{R}\in \kb{\R}}\Pi_{R\in \Mc_{\kb{R}}}l_{R},
\vspace{-0.7ex}\]
where $\Mc_{\kb{R}}$ consists of the relations from which $\kb{R}$ is mapped. 

\vspace{1ex}
Intuitively, $\usf(\Q,\kb{\R})$ quantifies the extent of queries in $\Q$
that can benefit from scan-free evaluation over $\kb{\R}$ %to reduce
by reducing \get invocations.
Furthermore, $\daccess(\Q, \kb{\R})$ further captures the
amount of data access of each \get operation over $\kb{\R}$.
Minimizing both measures together leads to \bdss over which $\Q$
has the best evaluation performance. Moreover,
$\size(\Q, \kb{\R})$ adds the dimension of
storage cost of $\kb{\R}$ and $\update(\Q, \kb{\R})$ captures incremental
maintenance cost \wrt the update frequencies of $\R$. Each of the
four measures is essential and {\em independent}, \ie~no one
can be expressed by others.
%\looseness = -1


\vspace{0.6ex}
Note that the smaller the measures are, the better
$\kb{\R}$ is for $\Q$. Hence, we aim to find $\kb{\R}$ while
minimizing the measures.



We aggregate the four measures
with a rank function:\looseness=-1

\vspace{-0.7ex}
\mat{2ex}{
  $f(\Q, \ak \kb{\R})$ = \= $c_{1}*\usf(\Q, \ak \kb{\R}) \ak + \ak c_{2}*\daccess(\Q,
  \ak \kb{\R})$ \ak + \\
  \> \hspace{2ex}
  $\ak c_{3}*\size(\Q, \ak \kb{\R}) \ak + \ak c_{4}*\update(\Q, \ak \kb{\R})$,
}
\vspace{-0.7ex}

\noindent
where coefficients $c_{1}$, \ldots, $c_{4}$ are given by 
users and are normalized, \ie $\textstyle\sum_{i=1}^{4}c_{i} = 1$.
That is,  $f(\Q,\ak \kb{\R})$ ranks normalized \bdss $\kb{\R}$ for $\Q$ by 
aggregating their scores of the four measures.




\begin{example}\label{exa-measures}
Recall $\kb{\R}_{1}$ and $\kb{\R}_{2}$ from Example~\ref{exa-mapping}.
Let $\Q$ consist of only $Q_{1}$ from Example~\ref{exa-baav}.
Assume $w_{Q} = 10$ and $l_{R} = 2$ for each relation $R$ in
$\R_{1}$. Consider rank function $f_{0}(\Q, \kb{\R})$ with $c_{1} = 0.98$,
$c_{2} = c_{3} = 0.01$ and $c_{4} = 0$.
%
Then $\sf(Q,\ak \kb{\R}_{1}) \ak = \ak 1$ and hence $\usf(\Q, \ak
\kb{\R}_{1}) \ak = \ak 0$;
$\daccess(\Q, \ak \kb{\R}_{1}) \ak = \ak 3$;
$\size(\Q, \ak \kb{\R}_{1}) \ak = \ak 7$; and
$\update(\Q, \ak \kb{\R}_{1}) \ak = \ak 6$.
Hence $f_{0}(\Q, \kb{\R}_{1}) = 0.1$.

\vspace{0.6ex}
For $\kb{\R}_{2}$, we have that $\usf(\Q, \ak
\kb{\R}_{2})\ak =\ak 0$, $\daccess(\Q, \ak \kb{\R}_{2}) \ak = \ak
5$, $\size(\Q, \ak \kb{\R}_{2}) \ak = \ak 13$, and $\update(\Q,
\ak \kb{\R}_{2}) \ak = \ak 12$. Hence, $f_{0}(\Q, \kb{\R}_{2}) =
0.18$, \ie $\kb{\R}_{2}$ is worse than $\kb{\R}_{1}$ under
$f_{0}$ (indeed, $\kb{\R}_{1}$ is superior under all aggregate rank functions). 

\vspace{0.6ex}
Now consider a revision $\kb{\R}_{3}$ of $\kb{\R}_{2}$ without
%Let $\kb{\R}_{3}$ be a revision of $\kb{\R}_{2}$ without
$\kb{\at{T}}_{1}$. Then
%one can verify that
$Q_{1}$ is not
scan-free over $\kb{\R}_{3}$ and hence $\usf(\Q,\ak \kb{\R}_{3})
= 10$; $\daccess(\Q, \ak \kb{\R}_{3}) = 4$; 
$\size(\Q, \ak \kb{\R}_{3}) \ak = \ak 10$; and
$\update(\Q, \ak \kb{\R}_{3}) \ak = \ak 10$.
Then $f_{0}(\Q,\ak \kb{\R}_{3})\ak =\ak 9.94$, \ie $\kb{\R}_{3}$
is ranked lower than $\kb{\R}_{2}$ by $f_{0}$.
\end{example}

\vspace{-0.4ex}




\stitle{The problem of \bds selection}.
The {\em \baav schema selection problem} is to compute,
given a workload $\Q$ over $\R$ and an aggregate rank function
$f$ (\ie $c_{1}$, \ldots, $c_{4}$),
the optimal \bds $\kb{\R}$ for $\Q$ based on $f$, \ie a \bds
$\kb{\R}$ for $\Q$ with the minimum $f(\Q, \kb{\R})$ among all \bdss
that are in the normal form for $\Q$.


\stitle{Complexity}. Its  decision problem is to decide, given $\Q$, $f$ and a
number $h$, whether there exists a \bds $\kb{\R}$ for
$\Q$ such that $f(\Q, \kb{\R})\leq h$.
We next show that the problem is intractable.
  Here an \SPC query 
  is {\em self-join free} if it has no self-join.


\begin{theorem}\label{thm-complexity}
  The \baav schema selection problem
  is \NP-complete for \warn{\SQL} %\SPC
  queries. 
It remains \NP-hard even $\Q$ consists of self-join free
\SPC queries only \warn{and $c_{3} = c_{4} = 0$ in
%the aggregate rank function
$f$.}
\end{theorem}

%\warn{Extend it to \SQL instead of \SPC}
\vspace{-0.4ex}

\begin{proofS}
To verify the upper bound, we first prove a small model
property: for any \bds $\kb{\R}$, there exists a \bds $\kb{\R}'$
that consists of no more than $\cd{\Q}$ \bss, each of size
bounded by $|\R|$, such that $f(\Q, \kb{\R}) \geq f(\Q,
\kb{\R}')$. Based on this, we give an \NP algorithm for selecting
  \bds that works as follows:
it first guesses a set $\kb{\R}$ of \bss with attributes from
$\R$ such that $\kb{\R}$ contains at most $\cd{\Q}$ \bss,
each of size at most $|\R|$;
it then checks whether $f(\Q, \kb{\R})\leq h$ and returns
$\kb{\R}$ if so.

The lower bound is verified by
reduction from the minimum set cover problem, which is
\NP-complete (cf.~\cite{GaJo79}).
The reduction uses $\Q$ with a single
self-join free \SPC query and only measures (1) and (2) in the
aggregate rank function $f$.
\end{proofS}


