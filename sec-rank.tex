\vspace{2ex}
%%%%%%%% Section 3 %%%%%%%%
\section{Ranking Cache Schemas}
\label{sec-rank}


We next develop criteria for ranking \bdss. We then
formulate the \bds selection problem  and
prove its intractability.


\warn{\stitle{Scan-free evaluability}.
One naturally wants to find \bdss that can make as many queries
scan-free as possible, to benefit from the feature of scan-free evaluation
from \baav databases.  However, this alone does not make good \bdss.
For instance, recall query $Q_{1}$ and \bss $\kb{\at{T}}_{1}$
and $\kb{\at{T}}_{1}'$ from Example~\ref{exa-baav}; $Q_{1}$ is
scan-free over both $\kb{\at{T}}_{1}$ and $\kb{\at{T}}_{1}'$
while its scan-free plan over $\kb{\at{T}}_{1}$ is superior than
that over $\kb{\at{T}}_{1}'$ since it fetches fewer attributes
and hence accesses fewer values. Moreover, making all queries
scan-free evaluable may not always be possible unless we use
\bss with small or even empty key attribute sets and large value
attribute sets, which basically fall back to table scans as the
conventional tuple-as-a-value model does.}


\vspace{0.6ex}
\warn{This motivates us to consider the {\em degrees} of \bss
below to further measure the quality of \bdss,
 % in addition to scan-free evaluability,
to distinguish good \bdss from bad ones although over both
queries can be scan-free.}


\warn{\stitle{Degree}.
The degree of a \bs $\kb{R}\bschema{X}{Y}$, denoted by
$\deg(\kb{R})$, is the size $|Y|$ of $Y$, \ie the number of
attributes in $Y$. The degree $\deg(\kb{\R})$ of a \bds
$\kb{\R}$ is as the sum of the degrees of all \bss in $\kb{\R}$.}

\warn{Intuitively, we want to find \bss with smaller degrees
since \get operations over them fetch fewer attributes and hence
tend to access smaller blocks of values. For example, the
scan-free evaluation of $Q_{1}$ over $\kb{\at{T}}_{1}$ fetches
fewer \at{transaction} values than over $\kb{\at{T}}'_{1}$.}


\vspace{1ex}
\warn{Note that scan-free evaluability and \bs degrees are measures
unique to the \baav model, and have not been used in the
conventional database design or index and view selection.
They have to work together to measure the performance of query
evaluation over the selected \bdss. This however is not easy as
shown below.}

\warn{\eetitle{Intricacy I}. Indeed, maximizing scan-free evaluability
of the selected \bds $\kb{\R}$ (\ie maximizing the number of
queries that can be made scan-free evaluable over $\kb{\R}$)
tend to minimize the key attributes in the \bss (\ie $X$ in \bs
$\bschema{X}{Y}$) of $\kb{\R}$.  Since $\kb{\R}$ is in normal
form for $\Q$, the degree of $\kb{\R}$ would increase if we
reduce key attributes.  Indeed, optimizing both criteria is
already intractable, as shown below.}

\begin{lemma}\label{lem-criteria}
\warn{Given a self-join free \SPC query $Q$ over database schema $\R$
and an integer $k$, it is already \NP-hard to decide whether
there exists a \bds $\kb{\R}$ over which $Q$ is scan-free and $\deg(\kb{\R}) \leq k$.}
\end{lemma}

\warn{Here an \SPC query is a relational algebra query composed of
selection, project and Cartesian-product (or join) operators
only. \SPC queries are the simplest but mostly used class of
\SQL queries. A self-join free \SPC query is an \SPC query
without self-joins.}


\begin{proofS} \warn{This is proved by reduction from the minimum set
  cover (\msc) problem, which is \NP-complete
  (cf.~\cite{Papa1994}). Each set of the \msc\ instance is
  encoded by a relation atom of $Q$ and each set cover is encoded
  by a \bds over which $Q$ is scan-free.}
\end{proofS}

\warn{\eetitle{Intricacy II}. The other challenge of optimizing \bds
with scan-free evaluability comes from semantic query rewriting.
Indeed, via syntactic checking, a query $Q$ may not look
scan-free over a \bds $\kb{\R}$;
however, one may find a semantically equivalent query $Q'$ which
is easily verified scan-free over $\kb{\R}$. Indeed, one can
readily prove that it is already undecidable to decide whether
an \RA (relational algebra) query is scan-free over a \bds
$\kb{\R}$. Hence, finding an optimal \bds \wrt scan-free
evaluability is also undecidable for \RA. Here \RA queries are
\SQL queries without group-by and aggregation.}

\vspace{0.4ex}
\warn{To alleviate this, we adopt the effective syntax of scan-free
\SQL queries of \cite{VLDB19} such that we consider scan-free
queries that are covered by the syntax. The syntax can express
all scan-free \SQL queries up to equivalence and can be checked in
\PTIME. The effective syntax requires reasoning about the \bss
over the queries via chasing, in a way that to some extent is
similar to the implication analysis of functional dependency
(see full version \cite{full} for details).}


\warn{\stitle{Ranking \bds}. Having presented two unique ranking
criteria for \bdss and analyzed their intricacy, we are now
ready to give the ranking function we will use for \bds design.
Given database schema $\R$ and a set $\Q$ of parametric
queries over $\R$, we rank \bds $\kb{\R}$ that are in normal
form for $\Q$ via an aggregated ranking function:
\[f(\Q, \kb{\R}) = c_{1}*\usf(\Q, \ak \kb{\R}) + c_{2}*\deg(\kb{\R}) + \ak c_{3}*\size(\kb{\R}),\]
where $c_{1}$, $c_{2}$ and $c_{3}$ are importance coefficients
of the criteria given by the users and are normalized, \ie
$\sum_{i=1}^{3}c_{i} = 1$; and
\be
  \item ({\em scan-free evaluability}) $\usf(\Q, \kb{\R})$ measures the number of
queries in $\Q$ that are {\em not} scan-free evaluable over $\kb{\R}$;
  \item ({\em degree}) $\deg(\Q, \kb{\R})$ is the total degree of $\kb{\R}$;
    and
  \item ({\em size}) $\size(\kb{\R})$ measures the total size of \bss in $\kb{\R}$.
\ee
In particular, $\size(\kb{\R})$ is defined as the total number
of attributes in \bss of $\kb{\R}$, \ie
$\size(\kb{\R}) = \sum_{\bschema{X}{Y}\in \kb{R}}|XY|$.
Intuitively, it is an estimation of the size of \baav databases
of $\kb{\R}$ at the schema level; the smaller $\size(\kb{\R})$
is, the smaller the \baav databases of $\kb{\R}$.}


\warn{\stitle{Remark}. (1) We acknowledge that there are many
other conventional criteria for ranking indices and views that can
also be applied to measure \bdss, such as the database size and
incremental maintenance cost. However, to focus more on the new
challenges incited by the \baav model (\ie scan-free
evaluability and degrees), we only adopted the size of \baav
databases in $f(\Q, \kb{\R})$, measured by $\size(\kb{\R})$ at
the schema level. In doing so, we also align with the main
targeted application scenarios of \baav databases, \ie
analytical \SQL queries for SQL-over-NoSQL (key-value) systems.
This said, in practice one can incorporate more coarse-grained
criteria in $f(\Q, \kb{\R})$ based on the targeted applications.
Our algorithms to be presented shortly are generic and robust to
such extensions.


\sstab (2) One can also attach an importance weight $w_{Q}$ to
each parametric query $Q$ in $\Q$, representing \eg its
frequency of being instantiated in the workload. We remark that
all the results in the sequel remain unchanged for such weighted
version.}





\begin{example}\label{exa-measures}
Recall $\kb{\R}_{1}$ and $\kb{\R}_{2}$ from Example~\ref{exa-mapping}.
Let $\Q$ consist of only $Q_{1}$ from Example~\ref{exa-baav}.
Consider rank function $f_{0}(\Q, \kb{\R})$ with $c_{1} = 0.98$
and  $c_{2} = c_{3} = 0.01$.
%
Then $\usf(\Q, \ak \kb{\R}_{1}) \ak = \ak 0$ since $Q_{1}$ is
scan-free over $\kb{\R}_{1}$.
One can further verify that
$\daccess(\Q, \ak \kb{\R}_{1}) \ak = \ak 3$ and
$\size(\Q, \ak \kb{\R}_{1}) \ak = \ak 7$.
Hence $f_{0}(\Q, \kb{\R}_{1}) = 0.1$.

\vspace{0.6ex}
Similarly, for $\kb{\R}_{2}$ we have that
$\usf(\Q, \ak \kb{\R}_{2})\ak =\ak 0$,
$\daccess(\Q, \ak \kb{\R}_{2}) \ak = \ak 5$ and
$\size(\Q, \ak \kb{\R}_{2}) \ak = \ak 13$.  Hence,
$f_{0}(\Q, \kb{\R}_{2}) = 0.18$, \ie $\kb{\R}_{2}$ is worse than
$\kb{\R}_{1}$ under $f_{0}$. Indeed, one can readily verify that
$\kb{\R}_{1}$ is superior under $f(\Q, \kb{\R})$ with all
possible coefficients $c_{1}$, $c_{2}$ and $c_{3}$.

\vspace{0.6ex}
Now consider a revision $\kb{\R}_{3}$ of $\kb{\R}_{2}$ without
$\kb{\at{T}}_{1}$. Then
$Q_{1}$ is not scan-free over $\kb{\R}_{3}$ and hence
$\usf(\Q,\ak \kb{\R}_{3}) = 1$. Further, observe that
$\daccess(\Q, \ak \kb{\R}_{3}) = 4$ and
$\size(\Q, \ak \kb{\R}_{3}) \ak = \ak 10$.
Hence $f_{0}(\Q,\ak \kb{\R}_{3})\ak =\ak 1.12$,
\ie $\kb{\R}_{3}$ is ranked lower than $\kb{\R}_{2}$ by $f_{0}$.
\end{example}

\vspace{-0.4ex}




\stitle{The problem of \bds selection}.
The {\em \baav schema selection problem} is to compute, given a
workload $\Q$ of parametric queries over $\R$ and an aggregate
rank function $f$ (\ie $c_{1}$, $c_{2}$ and $c_{3}$), the
optimal \bds $\kb{\R}$ for $\Q$ based on $f$, \ie a \bds
$\kb{\R}$ for $\Q$ with the minimum $f(\Q, \kb{\R})$ among all
\bdss that are in the normal form for $\Q$.


\stitle{Complexity}. Its  decision problem is to decide, given $\Q$, $f$ and a
number $h$, whether there exists a \bds $\kb{\R}$ for
$\Q$ such that $f(\Q, \kb{\R})\leq h$.
We next show that the problem is intractable.
  Here an \SPC query 
  is {\em self-join free} if it has no self-join.


\begin{theorem}\label{thm-complexity}
  The \baav schema selection problem
  is \NP-complete for \SQL %\SPC
  queries. 
It remains \NP-hard even $\Q$ consists of self-join free
\SPC queries only and $c_{3} = 0$ in
the aggregate rank function $f$.
\end{theorem}

%\warn{Extend it to \SQL instead of \SPC}
\vspace{-0.4ex}

\begin{proofS}
\warn{The \NP-hardness follows from Lemma~\ref{lem-criteria}.}
To verify the upper bound, we first prove a small model
property: for any \bds $\kb{\R}$, there exists a \bds $\kb{\R}'$
that consists of no more than $\cd{\Q}$ \bss, each of size
bounded by $|\R|$, such that $f(\Q, \kb{\R}) \geq f(\Q,
\kb{\R}')$. Based on this, we give an \NP algorithm for selecting
  \bds that works as follows:
it first guesses a set $\kb{\R}$ of \bss with attributes from
$\R$ such that $\kb{\R}$ contains at most $\cd{\Q}$ \bss,
each of size at most $|\R|$;
it then checks whether $f(\Q, \kb{\R})\leq h$ and returns
$\kb{\R}$ if so.
%
% The lower bound is verified by
% reduction from the minimum set cover problem, which is
% \NP-complete (cf.~\cite{GaJo79}).
% The reduction uses $\Q$ with a single
% self-join free \SPC query and only measures (1) and (2) in the
% aggregate rank function $f$.
\end{proofS}


