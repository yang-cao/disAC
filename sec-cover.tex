\vspace{2ex}
%%%%%%%% Section 5 %%%%%%%%
\section{Computing Optimal Schema Cover}
\label{sec-cover}

We next develop an algorithm to construct a universe set
$\kb{\A}$ of \bss for $\Q$, as part of the input for algorithm
\opts above. %in Section~\ref{sec-select}.

\stitle{Algorithm}. The algorithm, denoted by \usc, takes as input a workload $\Q$
over a database schema $\R$, and returns a set $\kb{\A}$ of \bss
for $\Q$.
While there exist up to exponentially many distinct \bss for
$\Q$, \usc computes $\kb{\A}$ with the following properties:
\bi
\item[(a)] for each K-FK join \SPC query $Q\in \Q$, any rank aggregate function $f$,
  and any \bds $\kb{\R}_{Q}$ that is in the normal form for $Q$,
  if $\kb{\R}_{Q}$ is optimal under $f$, then
  $\kb{\R}_{Q}\subseteq \kb{\A}$; and
\item[(b)] the time for constructing $\kb{\A}$ is polynomial in
  $|\Q|$ and $|\R|$.
  %\footnote{We assume that the number joins in each $Q\in \Q$ is bounded by a constant.}
\ei
By (b), the size of $\kb{\A}$ is also polynomial in
$|\Q|$ and $|\R|$.

To do this, \usc constructs $\kb{\A}$ for $\Q$ by
constructing a universe set $\kb{\A}_{Q}$ of \bss for each
$Q$ in $\Q$,
such that under any aggregate rank function $f$, the optimal
normalized \bds $\kb{R}_{Q}$ for $Q$ is %must be
subsumed by $\kb{\A}_{Q}$.
It returns the union of $\kb{\A}_{Q}$ for all $Q$ in $\Q$ as
the universe set $\kb{\A}$ of \bss for $Q$.
%
When $Q$ is an \SQL, we decompose it into its \SPC
sub-queries, and $\kw{\A}$ is the union of the universe sets of
  all the sub-queries.
Hence, below we assume that $\Q$ consists of \SPC queries.
\looseness = -1

More specifically, \usc generates $\kb{\A}_{Q}$ for each $Q$ as follows.

\eetitle{Step (1)}. It first generates two \bss for each $Q$:
$\kb{R}_{Q}^{1} = \bschema{P_{Q}}{Y_{Q}}$ and $\kb{R}_{Q}^{2} =
\bschema{P_{Q}Y_{Q}}{\emptyset}$. Here $\{\kb{R}_{Q}^{1}\}$ is
an optimal \bds for $Q$ when the scan-free evaluability (measure
(1))
and the size (measure (3)) are considered; similarly,
$\{\kb{R}_{Q}^{2}\}$ is optimal when degrees (measure (2)) is
concerned.
One can verify that $\kb{R}_{Q}^{1}$ and $\kb{R}_{Q}^{2}$ taken
together can subsume optimal \bdss for $Q$ under any ranking
function $f$ with $c_{4} = 0$.
\looseness = -1

\vspace{0.36ex}
Here we include $\kb{R}_{Q}^{2}$ only to guarantee property (a)
of $\kb{\A}$ claimed, \ie $\kb{\A}_{Q}$ can cover all optimal
\bdss for $Q$, even when only measure (2) is concerned alone. In
practice, we omit $\kb{R}_{Q}^{2}$ since measure (2) is typically
used along with other measures, \eg measure (1), to quantify
query evaluation cost over $\kb{\R}$.


\vspace{0.36ex}

\eetitle{Step (2)}. It then decomposes $\kb{R}_{Q}^{1}$ over relations of
$Q$, yielding \bss that do not span across multiple relations. The
process starts with the relation $R$ that contains the maximum
number of parameters of $Q$ (\ie $P_{Q}$). It generates
$\kb{R}_{Q} = \bschema{P_{Q}^{R}}{Y_{Q}^{R}}$ from $R$, where
$P_{Q}^{R}$ consists of attributes in $P_{Q}$ that are also in
$R$; similarly for $Y_{Q}^{R}$; it marks $R$ as {\em processed}.
It then identifies the relation $R'$ in $Q$ such that
  it contains the maximum
number of attributes that are in \bss $\kb{R}_{Q}$ with $R$
marked processed, and handles $R'$ along the same lines. The
process continues until all relations in $Q$ are marked
processed. This derives one \bs $\kb{R}_{Q}$ for each $R$ of $Q$
from $\kb{R}_{Q}^{1}$ generated in step (1). \looseness = -1

\vspace{0.36ex}

\eetitle{Step (3)}. This final step is to generate cross-relation \bss
for $Q$. For each set $\Sigma$ consisting of two relations in $Q$
that can be joined together with K-FK joins, \usc replaces both
relations with the universal relation $R_{\Sigma}$
(see~\cite{AbHuVi1995}) of $\Sigma$, and re-does step (2) over
$R_{\Sigma}$ and those relations that are not in $\Sigma$. It
generates again one \bs per each relation of $Q$ that is not yet
in $\Sigma$ and one \bs $\kb{R}_{Q}^{\Sigma}$ for $R_{\Sigma}$. Note
that $\kb{R}_{Q}^{\Sigma}$ may span across two relations in $\Sigma$.
The step is done %finishes
when all sets $\Sigma$ for $Q$ are processed
and the new \bss are all added to $\kb{\A}_{Q}$. \looseness = -1



\stitle{Implementing \usc}.
Algorithm \usc is outlined as Algorithm~\ref{alg-usc}.
Step (1) is carried out by lines~\ref{usc-l1}-\ref{usc-l2}.
followed by
  steps (2) and (3) (lines~\ref{usc-l3}-\ref{usc-l5}).
In particular, when $\Sigma$
contains only a single relation of $Q$, a single execution
  of lines~\ref{usc-l4}-\ref{usc-l5} conducts step (2);
the other iterations of lines~\ref{usc-l3}-\ref{usc-l5} are for 
step (3). The key part of each iteration is the decomposition of
$\kb{R}_{Q}^{1}$ computed in step (1) (line~\ref{usc-l2}) over
relations of $Q$, yielding ``projected'' \bss of
$\kb{R}_{Q}^{1}$, in various sizes. This is implemented via
procedure \decompose, also shown
in Algorithm~\ref{alg-usc} (lines~\ref{usc-l8}-\ref{usc-l15}), which
largely follows the description of steps (2) and (3) of above.
It takes the set of all \bss generated in steps (1)-(3) for
each $Q$ as $\kb{\A}_{Q}$ (lines~\ref{usc-l1}-\ref{usc-l5}) and
returns the union of $\kb{\A}_{Q}$ for all queries in $\Q$ as the
final universe set $\kb{\A}$ (lines~\ref{usc-l6}-\ref{usc-l7}).

%\warn{Check the highlighted part}

\begin{example}\label{exa-usc}
Recall $Q_{1}$ from Example~\ref{exa-baav}.
Step (1) of \usc generates $\kb{R}_{Q_{1}}^{1} \ak =\ak
\bschema{(\at{date}, \ak \at{payee})}{(\at{cid}, \ak \at{bid},
  \ak \at{city})}$ and 
$\kb{R}_{Q_{1}}^{2} = \bschema{(\at{date}, \ak \at{payee},
  \at{cid}, \ak \at{bid}, \ak \at{city})}{\emptyset}$. 
Step (2) of \usc populates $\kb{R}_{Q_{1}}^{1}$ into 
$\kb{\at{B}}_{1}$, $\kb{\at{T}}_{1}$ and $\kb{\at{C}}_{1}$ of
$\kb{\R}_{1}$ in Example~\ref{exa-mapping}.
Step (3) further generates $\kb{\at{TC}}_{2}\bschema{(\at{data},\ak
  \at{payee})}{(\at{cid},\ak \at{bid})}$ and 
$\kb{\at{CB}}\bschema{\at{cid}}{(\at{bid},\ak \at{city})}$.
Finally, \opts returns $\kb{\A}$ consisting of 
$\kb{\at{B}}_{1}$, $\kb{\at{C}}_{1}$, $\kb{\at{T}}_{1}$,
$\kb{\at{TC}}_{2}$, $\kb{\at{CB}}$.
Note that $\kb{\A}$ subsumes the optimal \bds $\kb{\R}_{1}$ and
$\kb{\R}_{2}$ of Example~\ref{exa-ILP-I}.
\end{example}

\vspace{-0.7ex}


\stitle{Analysis}.
Step (1) of \usc warrants property (a) of $\kb{\A}$ claimed at
the beginning of the section. Indeed, for each $Q$, step (1) of
\usc guarantees that $\kb{\A}_{Q}$ subsumes optimal \bdss under
any ranking function $f$ with $c_{4} = 0$. The key observation
here is that $\kb{R}_{Q}^{1}$ constructed in step (1) already forms an
optimal \bds under both measures (1) and (3).
After expanding $\kb{\A}_{Q}$ with \bss over individual relations
in $Q$ generated by step (2), by the definition of measure (4),
\usc further guarantees that all optimal \bdss for $Q$ are subsumed by
$\kb{\A}_{Q}$, even all measures are taken into account in $f$.
The analysis is based on the assumption that a \bds $\kb{\R}$ is
in the normal form for $Q$ if and only if for each \qcs of $Q$, there
exists \bs $\kb{R}$ in $\kb{\R}$ that supports $W$ (recall
Section~\ref{sec-select}); this holds when $\Q$
consists of K-FK join \SPC queries.
\looseness = -1

\vspace{0.36ex}

To see that property (b) also holds,
% for $\kb{\A}$,
observe that \usc computes $\kb{\A}$ with
no more than $2\cdd{\Q} + \frac{(\kappa+1)(\kappa+2)}{2}$
many \bss in $O(\kappa^{2}\cd{\Q})$-time, where $\kappa$ is the
maximum number of K-FK joins that a single query in $\Q$
contains. Indeed, step (1) of \usc generates $2\cdd{\Q}$ \bss in
$O(\cd{\Q})$-time; steps (2) and (3) generate no more than
$\frac{(\kappa+1)(\kappa+2)}{2}$ \bss in
$O(\frac{\cd{\Q}(\kappa+1)(\kappa+2)}{2})$-time, which is bounded
by $O(\kappa^{2}\cd{\Q})$.


\begin{myfloat}[t]
\vspace{1.2ex}
\begin{minipage}{0.50\textwidth}
  \removelatexerror
%%%%%%%%%%%%%%%%%%%%% Algorithm: USC
{\scriptsize
\setlength{\floatsep}{0cm} % set blank space above the figure
\setlength{\textfloatsep}{-2cm}% set blank space below the figure
\IncMargin{1em}
\vspace{-0.7ex}
\begin{algorithm}[H]
\setstretch{1.1}
%\SetAlgoNoLine
\Indentp{-2ex}
%\DontPrintSemicolon
%\SetAlgoLined
\KwIn{A database schema $\R$ and a workload $\Q$ over $\R$.}
\KwOut{A set $\kb{\A}$ of \bss for $\Q$.}
\Indentp{1em}
\BlankLine
\ForEach{$Q\in \Q$\label{usc-l1}}{$\kb{\A}_{Q} \la
  \{\kb{R}_{Q}^{1} = \bschema{P_{Q}}{Y_{Q}}), \kb{R}^{2}_{Q} =
  \bschema{P_{Q}Y_{Q}}{\emptyset})\}$\;\label{usc-l2}
  \ForEach{$\Sigma\subseteq \Sigma_{Q}$ consisting no more than
    2 relations\label{usc-l3}}{
    \tcc{$\Sigma_{Q}$: the set of relation atoms appeared in $Q$}
   % \tcc{relations in $\Sigma$ can be K-FK joined into a universal relation $R_{\Sigma}$}
    $\kb{\A}' \la \decompose(\Sigma_{Q}\setminus \Sigma \cup
    \{R_{\Sigma}\}, Q)$\label{usc-l4}\;
    $\kb{\A}_{Q} \la \kb{\A}_{Q}\cup \kb{\A}'$\label{usc-l5}\;
  }
}
$\kb{\A} \la \bigcup_{Q\in \Q} \kb{\A}_{Q}$\label{usc-l6}\;
\Return $\kb{\A}$\label{usc-l7}\;


\nonl\SetKwFunction{proc}{\pdecompose}
\nonl\SetKwProg{myproc}{Procedure}{}{}
\vspace{0.3cm}
\nonl\SetKwFunction{pdecompose}{\decompose}
% \setcounter{AlgoLine}{0}
\nonl \myproc{\pdecompose{$\R_{Q}, Q$}}{
  $\kb{\A} \la \emptyset$\;\label{usc-l8}
  \While{$\R_{Q}\neq \emptyset$}{
    \tcc{$\atset(R)$ (resp. $\atset(\kb{\A})$): the set of
      attributes in $R$ (resp. $\kb{\A}$)}
    $R \la \argmax_{R\in \R_{Q}}|\atset(R)\cap (P_{Q}\cup
    \atset(\kb{\A}))|$\;
    $X_{R}\la \atset(R)\cap (P_{Q}\cup \atset(\kb{\A}))$\;
    $Y_{R}\la \atset(R)\cap (Y_{Q}\cup \atset(\kb{\A}))$\;
    add \bs $\bschema{X_{R}}{Y_{R}}$ to $\kb{\A}$\;
    remove $R$ from $\R_{Q}$\;
  }
  \Return $\kb{\A}$;\label{usc-l15}
}
\caption{Algorithm \usc\label{alg-usc}} 
\end{algorithm}
\DecMargin{1em}
}
%%%%%%%%%%%%%%%%%%%%%
\end{minipage}
\vspace{-2.4ex}
\end{myfloat}

\eetitle{Remark}.
Note that step (3) of \usc only generates \bss over a single
relation or across two relations connected via K-FK join. While
this already guarantees property (a) of \usc, we can further
improve the coverage of the universe set $\kb{\A}$ returned by
\usc by covering all \bss across more than two relations for each
$Q$. To do this, we only need to lift the size restriction on set
$\Sigma\subseteq \Sigma_{Q}$ in line~\ref{usc-l3} of \usc, so
that it iterates over all subsets of $\Sigma_{Q}$. Note that
this only increases
the complexity of \usc to
$O(2^{\kappa}\cd{\Q})$-time; where $\kappa$ %$2^{\kappa}$
is typically a
small constant, \eg on average $2$ %$2^{2}$
for \tpch and \tpcds queries.

\eat{%EAT
  \begin{myfloat}[t]
\vspace{1.2ex}
\begin{minipage}{0.50\textwidth}
  \removelatexerror
%%%%%%%%%%%%%%%%%%%%% Algorithm: USC
{\scriptsize
\setlength{\floatsep}{0cm} % set blank space above the figure
\setlength{\textfloatsep}{-2cm}% set blank space below the figure
\IncMargin{1em}
\vspace{-0.7ex}
\begin{algorithm}[H]
\setstretch{1.1}
%\SetAlgoNoLine
\Indentp{-2ex}
%\DontPrintSemicolon
%\SetAlgoLined
\KwIn{Database schema $\R$ and workload $\Q$ over $\R$.}
\KwOut{A set $\kb{\A}$ of \bss for $\Q$.}
\Indentp{1em}
\BlankLine
\ForEach{$Q\in \Q$\label{usc-l1}}{$\kb{\A}_{Q} \la
  \{\kb{R}_{Q}^{1} = \bschema{P_{Q}}{Y_{Q}}), \kb{R}^{2}_{Q} =
  \bschema{P_{Q}Y_{Q}}{\emptyset})\}$\;\label{usc-l2}
  \ForEach{$\Sigma\subseteq \Sigma_{Q}$ consisting no more than
    2 relations\label{usc-l3}}{
    \tcc{$\Sigma_{Q}$: the set of relation atoms appeared in $Q$}
   % \tcc{relations in $\Sigma$ can be K-FK joined into a universal relation $R_{\Sigma}$}
    $\kb{\A}' \la \decompose(\Sigma_{Q}\setminus \Sigma \cup
    \{R_{\Sigma}\}, Q)$\label{usc-l4}\;
    $\kb{\A}_{Q} \la \kb{\A}_{Q}\cup \kb{\A}'$\label{usc-l5}\;
  }
}
$\kb{\A} \la \bigcup_{Q\in \Q} \kb{\A}_{Q}$\label{usc-l6}\;
\Return $\kb{\A}$\label{usc-l7}\;


\nonl\SetKwFunction{proc}{\pdecompose}
\nonl\SetKwProg{myproc}{Procedure}{}{}
\vspace{0.3cm}
\nonl\SetKwFunction{pdecompose}{\decompose}
% \setcounter{AlgoLine}{0}
\nonl \myproc{\pdecompose{$\R_{Q}$, $Q$}}{
  $\kb{\A} \la \emptyset$\;\label{usc-l8}
  \While{$\R_{Q}\neq \emptyset$}{
    \tcc{$\atset(R)$ (resp. $\atset(\kb{\A})$): the set of
      attributes in $R$ (resp. $\kb{\A}$)}
    $R \la \argmax_{R\in \R_{Q}}|\atset(R)\cap (P_{Q}\cup
    \atset(\kb{\A}))|$\;
    $X_{R}\la \atset(R)\cap (P_{Q}\cup \atset(\kb{\A}))$\;
    $Y_{R}\la \atset(R)\cap (Y_{Q}\cup \cup \atset(\kb{\A}))$\;
    add \bs $\bschema{X_{R}}{Y_{R}}$ to $\kb{\A}$\;
    remove $R$ from $\R_{Q}$\;
  }
  \Return $\kb{\A}$;\label{usc-l15}
}
\caption{Algorithm \usc\label{alg-usc}} 
\end{algorithm}
\DecMargin{1em}
}
%%%%%%%%%%%%%%%%%%%%%
\end{minipage}
\vspace{-2.4ex}
\end{myfloat}



\section{Computing Optimal Schema Cover}
\label{sec-cover}

We next develop an algorithm to construct a universe set
$\kb{\A}$ of \bss for $\Q$, as part of the input for \opts in
Section~\ref{sec-select}.

\stitle{Algorithm}. The algorithm, denoted by \usc, takes as input a workload $\Q$
over database schema $\R$, and returns a set $\kb{\A}$ of \bss
for $\Q$.
While there exist up to exponentially many distinct \bss for
$\Q$, \usc computes $\kb{\A}$ with the following properties:
\bi
\item[(a)] for each K-FK join \SPC query $Q\in \Q$, any rank aggregate function $f$,
  and any \bds $\kb{\R}_{Q}$ that is in the normal form for $Q$,
  if $\kb{\R}_{Q}$ is optimal under $f$, then
  $\kb{\R}_{Q}\subseteq \kb{\A}$; and
\item[(b)] the time for constructing $\kb{\A}$ is polynomial in
  $|\Q|$ and $|\R|$.
  %\footnote{We assume that the number joins in each $Q\in \Q$ is bounded by a constant.}
\ei
By (b), the size of $\kb{\A}$ is also polynomial in
$|\Q|$ and $|\R|$.

To do this, \usc constructs $\kb{\A}$ for $\Q$ by
constructing a universe set $\kb{\A}_{Q}$ of \bss for each $Q$ in $\Q$
such that, under any aggregate rank function $f$, the optimal
normalized \bds $\kb{R}_{Q}$ for $Q$ must be subsumed by $\kb{\A}_{Q}$.
It returns the union of $\kb{\A}_{Q}$ for all $Q$ in $\Q$ as
the universe set $\kb{\A}$ of \bss for $Q$. When $\Q$ contains
generic \SQL queries, it decomposes them into their \SPC
sub-queries. Hence, below we assume that $\Q$ consists of \SPC queries.

More specifically, \usc generates $\kb{\A}_{Q}$ for each $Q$ as follows.

\etitle{Step (1)}. It first generates two \bss for each $Q$:
$\kb{R}_{Q}^{1} = \bschema{P_{Q}}{Y_{Q}}$ and $\kb{R}_{Q}^{2} =
\bschema{P_{Q}Y_{Q}}{\emptyset}$. Indeed, $\{\kb{R}_{Q}^{1}\}$ is
an optimal \bds for $Q$ when scan-free evaluability (measure (1))
and size (measure (3)) are considered; similarly,
$\{\kb{R}_{Q}^{2}\}$ is optimal when data access (measure (2)) is
concerned.
One can verify that, $\kb{R}_{Q}^{1}$ and $\kb{R}_{Q}^{2}$ taken
together can subsume optimal \bdss for $Q$ under any ranking
function $f$ with $c_{4} = 0$.

Here we include $\kb{R}_{Q}^{2}$ only to guarantee property (a)
claimed of $\kb{\A}$, \ie $\kb{\A}_{Q}$ can cover all optimal
\bdss for $Q$, even when only measure (2) is considered alone. In
practice, we omit $\kb{R}_{Q}^{2}$ as measure (2) is typically
used along with other measures, \eg measure (1), to quantify
query evaluation cost over $\kb{\R}$.




\etitle{Step (2)}. It then decomposes $\kb{R}_{Q}^{1}$ over relations of
$Q$, yielding \bss that do not span multiple relations. The
process starts with the relation $R$ containing the maximum
number of parameters of $Q$ (\ie $P_{Q}$). It generates
$\kb{R}_{Q} = \bschema{P_{Q}^{R}}{Y_{Q}^{R}}$ from $R$, where
$P_{Q}^{R}$ consists of attributes in $P_{Q}$ that are also in
$R$; similarly for $Y_{Q}^{R}$; it marks $R$ as processed.
It then identifies the relation $R'$ in $Q$ that contains the maximum
number of attributes that are in \bss $\kb{R}_{Q}$ with $R$
marked processed, and process $R'$ along the same lines. The
process continues until all relations in $Q$ are marked
processed. This derives one \bs $\kb{R}_{Q}$ for each $R$ of $Q$
from $\kb{R}_{Q}^{1}$ generated in step (1).



\etitle{Step (3)}. Step (3) is to generate cross-relation \bss
for $Q$. For each set $\Sigma$ consisting of two relations in $Q$
that can be joined together with K-FK joins, \usc replaces both
relations with the universal relation $R_{\Sigma}$
(cf.~\cite{AbHuVi1995}) of $\Sigma$, and re-does step (2) over
$R_{\Sigma}$ and those relations that are not in $\Sigma$. It
generates again one \bs per each relation of $Q$ that is not in
$\Sigma$ and one \bs $\kb{R}_{Q}^{\Sigma}$ for $R_{\Sigma}$. Note
that $\kb{R}_{Q}^{\Sigma}$ may cross two relations in $\Sigma$.
The step finishes when all sets $\Sigma$ for $Q$ are processed
and the new \bss are all added to $\kb{\A}_{Q}$. 



\stitle{Implementing \usc}.
We next describe an implementation of algorithm \usc, shown as
Algorithm~\ref{alg-usc}. More specifically, it implements step
(1) via lines~\ref{usc-l1}-\ref{usc-l2}.
It then implements steps (2) and (3) via 
lines~\ref{usc-l3}-\ref{usc-l5}. In particular, when $\Sigma$
contains only a single relation of $Q$, this implements step (2);
the other iterations of lines~\ref{usc-l3}-\ref{usc-l5} implement
step (3). The key part of each iteration is the decomposition of
$\kb{R}_{Q}^{1}$ computed in step (1) (line~\ref{usc-l2}) over
relations of $Q$, yielding ``projected'' \bss of
$\kb{R}_{Q}^{1}$, in various sizes. This is implemented via
procedure \decompose also in Algorithm~\ref{alg-usc} (lines~\ref{usc-l8}-\ref{usc-l15}), which
largely follows the description of steps (2) and (3) of above.
It takes the set of all \bss generated in steps (1)-(3) for
each $Q$ as $\kb{\A}_{Q}$ (lines~\ref{usc-l1}-\ref{usc-l5}) and
returns the union of $\kb{\A}_{Q}$ for all queries in $\Q$ as the
final universe set $\kb{\A}$ (lines~\ref{usc-l6}-\ref{usc-l7}).

\begin{example}\label{exa-usc}
\end{example}




\stitle{Analysis}.
Step (1) of \usc warrants property (a) of $\kb{\A}$ claimed at
the beginning of the section. Indeed, for each $Q$, step (1) of
\usc guarantees that $\kb{\A}_{Q}$ subsumes optimal \bdss under
any ranking function $f$ with $c_{4} = 0$. The key observation
here is that $\kb{R}_{Q}^{1}$ constructed in step (1) already forms an
optimal \bds under both measures (1) and (3).
After expanding $\kb{\A}_{Q}$ with \bss over individual relations
in $Q$ generated by step (2), by the definition of measure (4),
\usc further guarantees that all optimal \bdss for $Q$ are subsumed by
$\kb{\A}_{Q}$, even all measures are taken into account in $f$.
The analysis is based on the assumption that a \bds $\kb{\R}$ is
in the normal form for $Q$ if and only if for each \qcs of $Q$, there
is \bs $\kb{R}$ in $\kb{\R}$ that supports $W$ (recall
Section~\ref{sec-select}); this holds when $\Q$
consists of K-FK join \SPC queries.
\looseness = -1

To see that property (b) also holds
for $\kb{\A}$. Observe that \usc computes $\kb{\A}$ consisting of
no more than $2\cdd{\Q} + \frac{(\kappa+1)(\kappa+2)}{2}$
many \bss in $O(\kappa^{2}\cd{\Q})$-time, where $\kappa$ is the
maximum number of K-FK joins that a single query in $\Q$
contains. Indeed, step (1) of \usc generates $2\cdd{\Q}$ \bss in
$O(\cd{\Q})$-time; steps (2) and (3) generate no more than
$\frac{(\kappa+1)(\kappa+2)}{2}$ many \bss in
$O(\frac{\cd{\Q}(\kappa+1)(\kappa+2)}{2})$-time, which is bounded
by $O(\kappa^{2}\cd{\Q})$.


\stitle{Remark}.
Note that step (3) of \usc only generates \bss over a single
relation or cross two relations connecting via K-FK join. While
this already guarantees property (a) of \usc, we can further
improve the coverage of the universe set $\kb{\A}$ returned by
\usc by covering all \bss cross more than two relations for each
$Q$. To do this, we only need to lift the size restriction on set
$\Sigma\subseteq \Sigma_{Q}$ in line~\ref{usc-l3} of \usc, so
that it iterates over all subsets of $\Sigma_{Q}$. Note that,
this only improves the complexity of \usc to
$O(2^{\kappa}\cd{\Q})$-time; where $2^{\kappa}$ is typically a
small constant, \eg as small as $2^{3}$ for \tpch and \tpcds
query workload.}%EAT
