\vspace{-1.7ex}
%%%%%%%% Section 1 %%%%%%%%
\section{Introduction}
\label{sec-intro}
%\vspace{-0.4ex}

Key-value (\kv) stores have been increasingly used in  storing
relations and answering \SQL queries for
\revise{data science and analytical applications}, %workloads},
as indicated by a variety of
open-source \SQL layers built on \kv stores, \eg Apache
Cassandra~\cite{cassandra}, Hive~\cite{hive},
Phoenix~\cite{phoenix}, SparkSQL~\cite{sparksql} and Kudu~\cite{kudu}.
\revise{The rationale behind
the practice is that \kv stores are often a natural
database storage in the Hadoop stack. This enables easy and
convenient integration with parallel computing frameworks such as
MapReduce, Spark and the aforementioned \SQL layers atop them, to
benefit from %harvest the power of
parallel computing in the cloud or over a  cluster %s
of commodity machines. After all, it is highly
impractical to require %ask
data scientists and analysts to configure and run a
parallel RDBMS, \eg Greenplum~\cite{greenplum}.}
%, for sometimes short lived data science or analytical applications.}

\vspace{0.36ex}
\revise{However, \kv stores were originally designed to target
workloads of large volumes of point read and write operations,
with a particular focus on horizontal scalability and
availability. Indeed,} they adopt a {\em
tuple-as-a-value} model \warn{for relations}:
a relation is \warn{conceptually} represented as key-value pairs
$(k, t)$, where $k$ is its tuple id or primary key $t$
is the entire tuple.
They support dictionary-like fast point access to retrieve tuples:
given a key $k$, a $\get(k)$ operation
locates the tuple
associated with $k$ in asymptotically constant time.
\revise{While this works well with point data access, the tuple-as-a-value
  model incurs
  %however causes large
  heavy scan overhead due to excessive \get
  invocations, which make %s
  it sub-optimal for analytical
workloads, \eg SQL.}


\begin{example}\label{exa-taav}
Consider query $Q_{1}$ to find ``cities with Walmart branches that
have customers still shopping at %on
Amazon on
June 1st, 2020''. It is
defined over
a database schema $\R_{1}$ with three relations
$\at{Walmart}(\at{bid},\ak \at{branch},\ak \at{city})$,
$\at{transaction}(\at{tid},\ak \at{cid},\ak \at{date},\ak
\at{payee})$ and $\at{customer}(\at{cid},\ak \at{gender},\ak
\at{address},\ak \at{bid})$. Written in \SQL, $Q_{1}$ is
%\warn{Change: R2O4}

\vspace{-0.1ex}
{\small
  \mat{4ex}{
    \SELECT\ \ \= \at{W.city}\\
    \FROM \> \at{Walmart} {\bf as} \at{W}, \at{transaction} {\bf as}
    \at{T}, \at{customer} {\bf as} \at{C}\\
    \WHERE \> \at{T.date} = {\normalsize 2020-06-01} \And
    \at{T.payee} ={\normalsize `Amazon'}\\
    \> \And \at{T.cid} =
    \at{C.cid} \And \at{C.bid} = \at{W.bid}
  }}
\vspace{-0.3ex}

Under tuple-as-a-value, \at{transaction} is stored in \kv-store
by using \at{tid} as the key attribute and all the other
attributes as the value part. Hence, to answer $Q_{1}$ one
has to scan \at{transaction} by fetch one tuple per \get;
similarly  for other relations. In other words, relation scans,
which are a common operation in \SQL evaluation, incur an excessive
number of \get invocations over relations in \kv stores.
\end{example}

%\vspace{0.6ex}
\revise{To alleviate the scan overhead for analytical \SQL processing over
\kv stores, a %the
{\em block-as-a-value} (\baav) model has recently been
proposed for representing relations in \kv-stores~\cite{VLDB19}}. The idea is to organize
relations in key-block pairs $(k, B)$ so that a set $B$ of values
is associated with key $k$. It allows us to assign arbitrary
attributes for $k$, not necessarily tuple id or primary keys.
Given such a key $k$, it fetches its block $B$ at once.
As shown in~\cite{VLDB19}, \baav helps us avoid relation scans;
under the tuple-as-a-value model, scan is implemented via
a sequence %of an excessive number
of $\get()$ operations without known keys,
and often fetches excessive irrelevant data. By fetching a block
of values at a time, \baav reduce these costly operations.



\begin{example}\label{exa-baav}
\revise{Continuing with Example~\ref{exa-taav},}
under \baav~\cite{VLDB19}, one can represent \at{transaction} as
$\kb{\at{T}}_{1}\bschema{(\at{date},\ak \at{payee})}{\at{cid}}$,
\at{customer} as $\kb{\at{C}}_{1}\bschema{\at{cid}}{\at{bid}}$,
and \at{Walmart} as $\kb{\at{W}}_{1}\bschema{\at{bid}}{\at{city}}$.
Then $Q_{1}$ can be answered using three \get operations
from such relations, without any table scans:
(a) It first fetches all $\at{cid}'s$ for ({\normalsize 2020-06-01},
{\normalsize `Amazon'}), via one \get over $\kb{\at{T}}_{1}$.
(b) It then fetches associated $\at{bid}'s$ for each
\at{cid} fetched in (a), via one \get over $\kb{\at{C}}_{1}$.
(c) It finally fetches all $\at{city}'s$ for $\at{cid}'s$
retrieved in (b), one \get~for %per
each \at{cid} over $\kb{\at{W}}_{1}$.
This illustrates the flexibility and benefits of the \baav
model: it allows us to retrieve only necessary data values for
answering $Q_{1}$ without the costly scans.\looseness=-1 %over \kv-stores.

\vspace{0.6ex}
However, the performance of $Q_{1}$ heavily relies on %the way
how
the relations are mapped into \baav. %the \baav forms.
Indeed, the
tuple-as-a-model schemas are a special case of the \baav
schemas, \revise{but under which} $Q_{1}$
has to %fully
scan all three relations. Even $Q_{1}$
can be made scan-free, its performance may still vary over
different \baav schemas. For example, assume that
we replace $\kb{\at{T}}_{1}$
with $\kb{\at{T}}_{1}'\bschema{\at{date}}{(\at{payee},\ak
  \at{cid},\ak \at{tid})}$, the %above
scan-free query %evaluation
plan above
still works but retrieves much %many
more data %values
from $\kb{\at{T}}_{1}'$:
\revise{%with $\kb{\at{T}}_{1}'$
it fetches \at{transaction}
records from all \at{payee}'s on 2020-06-01, instead of only
those from {\normalsize `Amazon'} when using $\kb{\at{T}}_{1}$.}
%\warn{Add more details: how serious   the extra retrieval is}
\end{example}

%\vspace{-0.3ex}

As shown in Example~\ref{exa-baav}, \baav supports {\em scan-free
query evaluation} provided with a good \baav schema.  However,
its performance highly depend on how the schema is designed.
While algorithms are already in place for %have been developed for
% checking and
evaluating %scan-free
queries over a \baav schema~\cite{VLDB19}, an open question is
how to design ``the best'' \baav schema?\looseness=-1

\vspace{0.36ex}

\eetitle{Challenges}. %In a nutshell,
Given a query load
$\Q$, we want to design ``the best'' \baav schema
$\kb{\R}$ that allows us to evaluate queries in $\Q$
with minimum time and space, \revise{by leveraging scan-free
\SQL evaluation provided by \baav.}
%without costly scans.
Here we consider parametric \SQL 
queries $Q[P_Q]$, where $P_Q$ is a set of attributes
that will be instantiated by users before execution.
Such queries are widely used in banks, supply chains and
Web search forms of e-commerce companies, among others.
Given a
set $\Q$ of parametric \SQL queries posed on an underlying
database schema $\R$, we aim to design a \baav schema
$\kb{\R}$ to store relations of $\R$ and 
answer queries of $\Q$ over \kv stores.

\vspace{0.36ex}
This schema selection problem is nontrivial.
For conventional database schema design, a variety of
normal forms \revise{are covered by textbooks and} provide us with
guidelines~\cite{AbHuVi1995}.
When it comes to \baav schema, no normal form is
yet in place. Several %A number of
questions have to be answered.
How to measure
the quality of a \baav schema $\kb{\R}$ for a query load $\Q$?
How to group key and value attributes for $\kb{\R}$
so that queries posed on underlying database of $\R$
can be answered over \kv stores while minimizing scans and
data retrieval, \revise{by optimally leveraging the scan-free
evaluability over \baav stores?}
How to reduce the space overhead of
storing relations of $\R$ as \baav pairs 
in a \kv system?
%How to reduce incremental maintenance cost
%of \kv stores in response to updates to the underlying
%databases of $\R$?
What is the complexity of this multi-criteria
optimization problem? Is it possible
%at all
to develop an efficient algorithm for schema
selection or is it beyond reach in practice?


%%%%%%%%
\stitle{Contribution \& organization}. This paper aims to investigate %tackles
these questions, and give a full treatment to the design of
\baav schemas for parametric \SQL queries over \kv-stores.

\eetitle{(1) A framework} (Section~\ref{sec-schema}). We propose a
framework to speed up parametric \SQL queries on top of \baav stores.
Underlying the framework is a module for selecting a \baav schema
and answering queries over the \baav database. %We also
Moreover, we introduce
a normal form of \baav schemas for parametric \SQL queries.

\eetitle{(2) Ranking \baav database schemas} (Section~\ref{sec-rank}).
We propose measures to rank \baav schemas for a
given parametric query workload. Based on these
we formulate the \baav schema selection problem.
We show that the problem is \NP-complete and remains \NP-hard
even for a single join query without self-joins.\looseness=-1

\eetitle{(3) Selecting optimal schemas} (Section~\ref{sec-select}).
We develop a method to select the best \baav  schema 
from any given set $\kb{\A}$ of \baav relation schemas, by reducing it to 
integer linear programming (ILP). Expressing all the ranking
criteria in ILP, the method computes an optimal \baav
schema by using existing efficient ILP solvers, and
ensures the scan-free evaluability in particular.
%\looseness = -1

\eetitle{(4) Computing schema cover} (Section~\ref{sec-cover}).
As part of the input of the method proposed in (3) above,
we %also
develop an algorithm that can efficiently construct a
set $\kb{\A}$ of \baav relation schemas, in polynomial
size, for a parametric query workload $\Q$.
%as part of the input of the method in (3) above.
It guarantees that for each $Q\in \Q$, under
any ranking function $f$, the optimal \baav schema
normalized for $Q$ is subsumed in $\kb{\A}$.
%\looseness = -1

Taken together, (3) and (4) provide a practical
  algorithm for schema selection that guarantees certain optimality.
  

\eetitle{(5) Experimental study} (Section~\ref{sec-expt}). 
Using \warn{\tpch~\cite{tpch} and \imdb~\cite{LeisGMBK015} benchmarks}, we
evaluate the effectiveness
of deduced \baav schema %by our algorithms
and the efficiency of
our schema selection algorithms.
We find the following.
(a) Scan-free evaluation %does
works well with \baav schemas found
by our algorithms. With SparkSQL-over-Cassandra (\soc),
on average \warn{\tpch and
  \imdb built-in benchmark queries} can be made \xx and \xx
times faster, respectively, over \revise{\baav schemas
%are
computed} by our algorithms than over the conventional
tuple-as-a-value schemas.
(b) \revise{Our algorithms are effective.
Over \tpch and \imdb,  \soc is 4.28 and \xx times
faster under \baav schemas generated by our algorithms than under  
those generated by the baselines, respectively.}
Moreover, the gaps increase when the queries have more joins or the datasets grow bigger.
% than relations stored under
% the conventional tuple-as-a-value model.
(b) Our algorithms are able to select
the optimal normalized \baav schema for \warn{\tpch and \imdb}
queries within 0.42 and 0.27 seconds, respectively.
\warn{Revise after new experiments.}

\vspace{0.6ex}
\revise{All formal details and proofs can be found in the full
version~\cite{full}.}

\stitle{Related work}. The related work is organized as follows.

\eetitle{\SQL over \kv-stores}.
\kv systems~\cite{AmazonKV, FacebookKV, cassandra, bigtable,
hbase, F1} were originally designed to store schemaless
\kv-stores (\ie collection of \kv pairs) and to provide low-level
fast point access to retrieve \kv records. Due to %Because of
their
superior horizontal scalability and availability with even
commodity machines, they have been increasingly used to store
relations and support scalable \SQL processing over
\kv-stores~\cite{spanner1,spanner,myrocks}. This is
\revise{implemented under}
%realized by
the SQL-over-NoSQL architecture~\cite{kvscan}, where relations
are stored in \kv-stores using the tuple-as-a-value model, and
\SQL processing is conducted by a separated \SQL
layer~\cite{cassandra,hive,phoenix,sparksql,drill,impala,kudu} on
top of the \kv-stores. Its research mainly focuses on the design
and optimization of \kv systems to improve their point access and
update performance.

Closer %to this work
is the recent work~\cite{VLDB19} on the data model used to
represent relations in \kv-stores. It observed that the
tuple-as-a-value model, which is the common practice of storing
relations in \kv systems, may incur excessive overhead of \get
invocations for \SQL evaluation due to %because of
frequent %table
scans
that do not fit  the point access design of \kv systems well.
It proposed the \baav model for relations in \kv-stores so as to
reduce scans for \SQL evaluation over relations in \kv-stores.

\vspace{0.6ex}
This work targets an open problem left by~\cite{VLDB19}, \ie
the design, \revise{ranking} and
selection of a good database schema for relations in \kv-stores
under the \baav model to support parametric query workloads; the
study is vital to making \revise{practical}
use of the \baav model in practice.
Taken together with the algorithms of~\cite{VLDB19} for
scan-free evaluation, it makes one step further toward speeding
up \SQL query answering on \kv systems under the
block-as-a-value model.


\eetitle{Index and view selection}.
\revise{To speed up get operations over non-key
attributes, modern \kv systems have explored %can
%attempt to use %secondary and covering
the use of} indices (see \eg~\cite{ChrysafisCDDEGG19}).
Consider a relation $D$ in \kv stores and
$\get(k)$ operations with $k$ on some non-key attributes $X$ of
$D$. \revise{To avoid scanning the entire $D$, 
the current approach is to build an index
on attributes $X$, which yields essentially a tuple-as-a-value
relation with $X$ padded with tuple identifiers of $D$
as a compound key~\cite{QaderCH18}. However, \get operations
with such an index still have to scan over the ``index table''
since $k$ is not unique for desired attributes of $X$}.
As noted in  \cite{VLDB19},
\baav is complementary to the indexing
approach, and one can combine the two by building indices over
\baav relations.

\vspace{0.6ex}
\revise{Related to this work is also index selection
(\eg~\cite{BrunoC05,BrunoCKNRS11,PapadomanolakisA07}).
For both RDBMS and SQL over \kv stores, index-based data access
is to fetch entire tuples (or tuple identifiers), and it is
typically carried out by greedy heuristics.
In contrast, \baav databases need to fetch and compose
values (\ie relevant partial tuples)  when answering queries.
The objective of \baav schema selection is to make best use of 
scan-free evaluability that is unique to \baav, and it
requires aligning and reasoning about \baav schemas. 
In light of these, the greedy heuristics for 
index selection cannot be applied to
\baav schema selection. While \cite{PapadomanolakisA07} also uses
ILP to select pre-defined indices, it reduces the task
to a minimum set cover (\msc) problem and adopts ILP
formulation of \msc directly. As will be seen in
Section~\ref{sec-rank}, \baav schema selection
requires reasoning about scan-free evaluability, which
is much more demanding than \msc}.


\vspace{0.6ex}
\revise{Also related is the study of materialized view
selection (see \cite{DBLP:journals/sigmod/MamiB12} for a
survey). It is to pick a set of queries and materialize their answers
in order to reuse them for future queries.
For the same reasons as for index selection above, view
selection is often carried out by greedy heuristics with
additional pruning strategies, which are unable to reason about \baav schemas
for scan-free evaluation.}

\eat{
\eetitle{Index and view selection}.
Modern \kv systems can also use %secondary and covering
indices to speed up get operations over non-key
attributes (see \eg~\cite{ChrysafisCDDEGG19}).
Consider relation $R$ in \kv stores and
$\get(k)$ operations with $k$ on some non-key attributes $X$ of
$R$. To avoid scanning the entire $R$,  one can build an index
on attributes $X$, which conceptually is a tuple-as-a-value relation with
$X$ padded with tuple identifiers of $R$ as a compound
key~\cite{QaderCH18}.  While it shares similar spirit and
objectives of \baav schemas,  \get operations with such an index
still incur scans over the ``index table'' as $k$ is not unique
in keys. As discussed in \cite{VLDB19}, \baav complements to the indexing
approach and one can combine the two by building indices over
\baav relations as well.

Index selection
(\eg~\cite{BrunoC05,BrunoCKNRS11,PapadomanolakisA07}) is also
related to the study of \baav schema selection; however, its
techniques cannot be applied to the latter. In particular,
index-based data access, in both RDBMS and SQL over \kv
settings, remains to fetch tuples (or tuple identifiers), while
values (\ie partial tuples) are fetched and composed when
answering queries over \baav databases~\cite{VLDB19}. Due to
this, designing \baav databases involves aligning and reasoning
about \baav schemas to leverage scan-free evaluability that is
unique to \baav.
%
By contrast, index selection typically relies on greedy
heuristics instead.  While \cite{PapadomanolakisA07} also uses
ILP to select pre-defined indices, it essentially treats the
task as a minimum set cover (\msc) problem and adopts ILP
formulation of \msc\ directly. As will be shown in
Section~\ref{sec-rank}, designing \baav schemas
involves reasoning about scan-free evaluability, which
requires machinery beyond that of \msc.

\vspace{0.6ex}
Similarly, related is also the work of materialized view
selection (see \cite{DBLP:journals/sigmod/MamiB12} for a
survey), which is to pick a set of queries so that we can
materialize their answers and reuse them for future queries
as much as possible. For the same reasons as
index selection above, view selection is often carried out by
greedy heuristics with additional pruning strategies, which
cannot capture the complications of reasoning with \baav schemas
for scan-free evaluation.}






\eetitle{Database design and normalization}.
Relational database design and normalization are classical topics
covered by all database textbooks, which are guided by a %number of
variety of
normal forms to reduce data redundancy and update anomalies
(see~\cite{DBSchema} for a survey). The design of NoSQL database
schemas has also recently been investigated by \eg
\cite{NoSQLschema1,NoSQLschema2}. Different from the
  prior work, we study
schema design and selection for relational databases in the
form of \kv-stores, while \cite{NoSQLschema1,NoSQLschema2} are
for the design of NoSQL databases instead of relations.
\looseness = -1




\eetitle{Bounded evaluation}.
Related to the concept of \baav schemas is also the study of
bounded evaluation~\cite{PODS14,PODS15,SIGMOD16,PODS16}, to
formalize scale independent query evaluation under access
constraints, which are, to some extent, a restricted form of the
\baav schema with cardinality constraints. %(cf.~\cite{VLDB19}).
%Indeed,
The techniques of bounded evaluation can be adopted to answer
queries over \baav databases as observed by \cite{VLDB19}.\looseness=-1

This work differs from bounded evaluation in that we focuses on
the selection and design of \baav schemas, while bounded
evaluation considers deciding scale independence of query
evaluation %under access constraints.
in the presence of cardinality constraints.
This said, the selection method for \baav schemas can be
  adapted to selecting access constraints, which has not been
  studied for bounded evaluation.


\eat{%EAT
  \eetitle{\SQL over \kv-stores}.
\kv systems~\cite{AmazonKV, FacebookKV, cassandra, bigtable,
hbase, F1} have been originally designed to store schemaless
\kv-stores (\ie collection of \kv pairs) and to provide low-level
fast point access to retrieve \kv records. Because of their
superior horizontal scalability and availability with even
commodity machines, they have been increasingly used to store
relations and support scalable \SQL processing over
\kv-stores~\cite{spanner1,spanner,myrocks}. This is realized by
the SQL-over-NoSQL architecture~\cite{kvscan}, where relations
are stored in \kv-stores using the tuple-as-a-value model and
\SQL processing is enabled by a separated \SQL
layer~\cite{cassandra,hive,phoenix,sparksql,drill,impala,kudu} on
top of the \kv-stores. The research mainly focuses on the design
and optimization of \kv systems to improve their point access and
update performance.

Closer is the recent work~\cite{VLDB19} on the data model used to
represent relations in \kv-stores. It observed that the
tuple-as-a-value model, which is the common practice of storing
relations in \kv systems, may cause excessive overhead of \get
invocations for \SQL evaluation because of frequent table scans
that do not well fit the point access design of \kv systems.
It proposed the \baav model for relations in \kv-stores so as to
reduce scans for \SQL evaluation over relations in \kv-stores.

Different from \cite{VLDB19}, we focus on the design, rank and
selection of a good database schema for relations in \kv-stores
to support parametric query workloads, which is vital to make use
of the \baav model in practice; instead, \cite{VLDB19} focuses on
the properties of the \baav model, \eg scan-free evaluation, and
the conditions of checking the properties.

\eetitle{Database design and normalization}.
Relational database design and normalization is a classic topic
covered by all database textbooks, which is guided by a number of
normal forms to reduce data redundancy and update anomalies
(see~\cite{DBSchema} for a survey). The design of NoSQL database
schemas has also recently been investigated by \eg
\cite{NoSQLschema1,NoSQLschema2}. Different from them, we study
the schema design and selection for relational databases in the
form of \kv-stores, while \cite{NoSQLschema1,NoSQLschema2} are
for the design of NoSQL databases instead of relations. 

\eetitle{Materialized views}. Related is also the work of view
selection for materialized views (see
\cite{DBLP:journals/sigmod/MamiB12} for a survey), which is to
pick a set of queries so that we materialize their answers and
re-use them for future queries to save computation as much as
possible. In some sense, the \baav databases can be viewed as a
special ``view'', where the view query is simply a relation in
the form of a \kv-store. Nonetheless, due to the very different
objectives and focus, conventional view selection techniques cannot carry
over to the design and select \baav schemas.

\eetitle{Bounded evaluation}.
Related to the concept of \baav schemas is also the study of
bounded evaluation~\cite{PODS14,PODS15,SIGMOD16,PODS16}, to
formalize scale independence of query evaluation under access
constraints, which are, to some extent, a restricted form of the
\baav schema with cardinality constraints (cf.~\cite{VLDB19}).
Indeed, the techniques of bounded evaluation can be adopted to answer
queries over the \baav databases as observed by \cite{VLDB19}.
This work differs from bounded evaluation in that we focuses on
the selection and design of \baav schemas while bounded
evaluation considers deciding scale independence of query
evaluation in the presence of cardinality constraints. 
}%EAT















\eat{%EAT
  \section{Introduction}
\label{sec-intro}

While key-value (\kv) systems have been originally designed to
support dictionary-like data access to retrieve and store large
datasets often referred to as \kv stores, they found increasingly
prevalent use in storing relations and supporting \SQL queries.
For example, a number of open-source \SQL layers have been built
on top of \kv systems to add the support of \SQL query evaluation
over \kv-stores, \eg Apache Cassandra~\cite{cassandra},
Hive~\cite{hive}, Phoenix~\cite{phoenix},
SparkSQL~\cite{sparksql}, Drill~\cite{drill},
Impala~\cite{impala} and Kudu~\cite{kudu}.

These \kv stores are implemented as key-value pairs that are
organized in a distributed hash table (DHT), which is a sparse
data structure that features fast point access via a \get
operation: given a key $k$, $\get(k)$ locates the value
associated with $k$ in asymptotically constant time. As the
common practice, relations are typically represented in \kv
stores using the {\em tuple-as-a-value} model, where each tuple
$t$ is represented as a $(k, t)$, where $k$ is the tuple id or
primary key of $t$ and $t$ is the entire tuple. Under such model,
relations are organized as key-tuple pairs that support fast
point access to retrieve a tuple given its key.
\looseness = -1

However, as observed in \cite{VLDB19}, scan over relations in \kv
stores under the tuple-as-a-value model is particularly costly
since it incurs excessive number of \get invocations to retrieve
tuples via point accesses. Meanwhile, scan is an inevitable and
frequently used operation when answering \SQL queries.

To tackle the this, a new model for representing relations in
\kv-stores, named {\em block-as-a-value} (\baav), has been proposed in
\cite{VLDB19}, to reduce the costly scans and the amount of data
access when evaluating \SQL queries over \kv-stores. The idea is to organize
relations in key-block pairs $(k, B)$ so that a set $B$ of values
is associated with key $k$. This allows us to assign any
attributes for $k$, not necessary only tuple id or primary keys.
As a result, we have more flexibility in representing relations
in \kv-stores, avoid the costly scans, and reduce the amount of
data accessed for answering \SQL queries. 


\begin{example}\label{exa-baav}
Illustrate the block-as-a-value model 
\end{example}

As shown in Example~\ref{exa-baav}, under the \baav
%block-as-a-value
model, one can allow scan-free query evaluation over \kv-stores,
which avoids the costly scans over \kv-stores and reduces the
amount of data values retrieved for answering \SQL queries. 

While algorithms for checking whether queries can be answered
over a \baav 
%block-as-a-value
schema have been developed in
\cite{VLDB19}, an open question is how to design the best
relational databases under this model. As illustrated in
Example~\ref{exa-baav}, schemas may lead varying performance of
the queries.

In this work, we answer the problem for a parametric
queries, which are widely by web applications to interact with
backend database systems with parameters instantiated during
execution. Given a set $\Q$ of parametric queries over database
schema $\R$, we aim to design the best \baav
%block-as-a-value
database schema of $\R$ for $\Q$, denoted by $\kb{\R}$,
such that queries instantiated from $\Q$ can be best supported by
$\kb{\R}$, and moreover, the cost of storing and maintaining
instances of $\kb{\R}$ is minimized. 

The problem is nontrivial. While it is related to \eg database
schema design and view selection in conventional relational
database systems, it is much more challenging. 
Apart from existing challenges in database design and view
selection in the conventional setting, among others we have to
answer the following questions:
(1) How to measure the quality of a database schema $\kb{\R}$
under the \baav model for $\Q$?
(2) How to pick and group key and value attributes for $\kb{\R}$
so that queries designed over $\R$ can be optimally answered over
the new \baav schema $\kb{\R}$?
(3) How to maintain and use $\kb{\R}$ to answer queries
instantiated from $\Q$ over original database schema $\R$?
As will be shown shortly in Section~\ref{sec-rank},
it is already \NP-hard to find the best
\baav schema for a single simple join query.



\stitle{Contribution}. We given a full treatment to the design of
\baav database schemas for parametric queries over \kv-stores.

\eetitle{(1) A framework} (Section~\ref{sec-schema}). We give a
framework to speed up parametric queries via \baav stores.
Underlying the framework is the module for designing and using a
sibling database under the \baav model for answering parametric
queries.

\eetitle{(2) Ranking \baav database schemas} (Section~\ref{sec-rank}).
We propose measures to rank \baav database schemas for a
given parametric query workload, based on which we formulate the
problem of \baav schema design. We show that the problem is
\NP-complete even for a single join query without self-joins.

\eetitle{(3) Selecting optimal schemas} (Section~\ref{sec-select}). 
We develop a method that can select the best \baav database
schema, \ie a set of \baav relation schemas, from any given
universal set of \baav relation schemas, by reducing to integer
linear programming (ILP). In particular, we express the scan-free
evaluability property for parametric queries and all the ranking
measures in ILP so that we can compute the optimal \baav database
schema by making use of existing efficient ILP solvers.
\looseness = -1

\eetitle{(4) Computing schema cover} (Section~\ref{sec-cover}).
We also develop an algorithm that can efficiently construct a
universe set $\kb{\A}$ of \baav relation schemas of polynomial
size for a parametric query workload $\Q$, as part of the input
of step (3) above. It guarantees that for each $Q\in \Q$, under
any ranking function $f$, the optimal \baav database schema
normalized for $Q$ is subsumed in $\kb{\A}$.
\looseness = -1

\eetitle{(5) Experimental study} (Section~\ref{sec-expt}). 
Using real-life datasets and standard database benchmarks, we evaluate
the effectiveness of our approach to \baav database schema design
for \kv-stores. 


\stitle{Related work}. The idea of database as a cache is related
to several strands of research categorized as follows.  To ADD.


\eat{%EAT
\eetitle{Parametric query evaluation}.
\eat{While there have been a
host of works on parametric query optimization, they focus on
deciding whether or when a precomputed query plan for one
instantiated query should be used for another query instantiated
from the same parametric
query~\cite{paraQ1,paraQ2,paraQ3,paraQ4,paraQ5,paraQ6,paraQ7}.
Instead, \tbc}



\eetitle{Database design}.
\eat{As a common practice, database design is typically guided by the
semantics of the datasets via, \eg entity relationship
diagrams~\cite{Raghu2000}. Once the database schema is
determined, it usually will not be altered due to its
implications on the applications built upon it. On the other
hand, the query workload that runs over the database is driven by
the database applications. While it is a folklore that the
database design has an evident impact on the query performance,
the disconnection between the database design and the query
workload leaves this largely unexploited, causing suboptimal
query performance. \tbc}


\eetitle{Caching}.
\eat{much less attention is paid to the conceptual modeling of cached
datasets \tbc}

\eetitle{View materialization}.


\eetitle{Bounded evaluation}.
\eat{The concept of cache schema is
quite related to the study of bounded
evaluation~\cite{PODS15,SIGMOD16,PODS16,VLDB17,VLDB19}. The idea
is to organize databases to provide a dictionary like data access
that fetches sets of distinct data values instead complete tuples
or columns. Combining with cardinality constraints on the
datasets, bounded evaluation can answer queries by accessing much
smaller and even bounded size of data, yielding considerable
speedups. \tbc}
}%EAT
}%EAT
