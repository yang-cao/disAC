\vspace{2ex}
%%%%%%%% Section 2 %%%%%%%%
\section{Block-as-a-Value Databases}
\label{sec-schema}


We first review the \baav model for relations in \kv-stores. We
then propose a normal form of \baav database schemas, and develop a
framework for speeding up parametric queries over \kv-stores with
normalized \baav databases.


\stitle{Parametric queries}.
Let $\R$ be a database schema, \ie a set of relation schemas.  
We consider parametric queries of the form $Q[P_{Q}]$, where
$Q$ is an \SQL query over $\R$, and $P_{Q}$ is a set of
attributes in $Q$ referred to as {\em parameters}, whose values will
be instantiated before %during
execution.
%Here we consider \SPC queries, \ie 
We denote by $Q[P_{Q} = \bar  c]$
an {\em instantiation of} $Q[P_{Q}]$
  by setting $P_{Q}$ to constants $\bar c$.

\vspace{0.6ex}
For instance, if we replace the constants ``{\normalsize
2020-06-01}'' and ``{\normalsize Amazon}'' in $Q_{1}$ of
  Example~\ref{exa-baav} with parameters (variables),
  %then
  it becomes a
parametric query and $Q_{1}$ is one of its instantiations.

\eat{%EAT
\vspace{0.6ex}
To simplify the discussion, below we focus on \SPC queries, \ie
relational algebra queries with selection ($\sigma$), project
($\pi$) and Cartesian-product ($\times$) or join ($\Join$); they
are the most common \SQL queries in practice.
As will be shown shortly, our methods can be readily extended to
support generic \SQL as well.
}%EAT

%%%%%%
\stitle{BaaV databases}. Similar to conventional relational
databases, a block-as-a-value~\cite{VLDB19} (\baav) database is
defined over a {\em \baav schema} $\kb{\R}$ (abbreviated
as \bds) that consists of {\em \baav relation schemas}
(\bss). Here a \bs $\kb{R}$ is of the form $\bschema{X}{Y}$,
where $X$ and $Y$ are sets of attributes. A \baav instance
$\kb{D}$ of $\kb{R}$ is a collection of key-block pairs $(k, B)$,
where each $k$ is an $X$-value and $B$ is a set of $Y$-values.
A {\em \baav database} $\kb{\D}$ of \bds $\kb{\R}$ is a
collection of \baav instances of \bss in $\kb{\R}$.
We refer to
the maximum size of the blocks in $\kb{D}$ as {\em the degree
of $\kb{D}$}.

\vspace{0.6ex}
A \baav instance $\kb{D}$ supports a \get operation such that given
an $X$-value as $k$, $\get(k)$ returns the block of
$Y$-values associated with $k$ in $\kb{D}$.
%
A bulk $\bget(T)$ operation with a set $T$ of $X$-values is a
sequence of $\get(k)$ when $k$ draws values from $T$.
  We use 
bulk $\bget(\kb{D})$ with $\kb{D}$ to denote a scan of \baav instance
$\kb{D}$; when it is clear from the context, we also write
$\bget(\kb{R})$ as a scan of the instance of \bs $\kb{R}$.
%\looseness = -1


\warn{\stitle{Scan-free plans}.}
A {\em query plan} $\xi$ over $\kb{\R}$ is a sequence
of either operations
from relational algebra or bulk $\bget(T)$ operations, where each $T$ is
(a) a constant, %in the queries,
(b) an intermediate result computed by a prior operation, or
(c) a \bs $\kb{R}$ (\ie a scan of a relation in \baav).
Plan
$\xi$ is called {\em scan-free} if it does not have any
$\bget(\kb{R})$
operation, \ie it involves no scans when executing over any \baav
database of $\kb{\R}$. It is called {\em strongly scan-free}
if moreover, 
the parameter $T$ of every $\bget(T)$ is a constant (\ie of type (a) only).

\vspace{0.6ex}
\warn{As observed in \cite{VLDB19},
% for the same query, over the same \baav database
 a scan-free plan can be orders of magnitude
faster than plans that are not scan-free.  Moreover, \baav
databases are a natural fit for \kv systems where key-block
pairs can be encapsulated as key-value pairs, so that existing
systems can easily benefit from \baav databases without modifications.}






\stitle{Schema mapping}.
Consider  a database schema $\R$.
A \baav database schema $\kb{\R}$ is called
a {\em sibling} of $\R$
if there exists an injective partial function $g$ that 
maps attributes of relations in $\R$ to those of \bss in
$\kb{\R}$ such that
(1) for each $\kb{R}\bschema{X}{Y}$ in $\kb{\R}$ and each
attribute $A\in XY$, $g^{-1}(A)$ is an attribute of of $\R$;
and (2) $g^{-1}(XY)$ consists of attributes either
(a) from the same relation schema in $\R$, or
(b) from multiple relation schemas $R_{1}$, \ldots, $R_{l}$ in
$\R$ such that their common attributes (\ie key-foreign
key join attributes) of $R_{1}$, \ldots, $R_{l}$ are also in $XY$.
In this case, $g$ is referred to as a {\em mapping} from $\R$ to
$\kb{\R}$.


\vspace{0.6ex}
A mapping $g$ from database schema $\R$ to \bds
$\kb{\R}$ naturally maps relational databases $\D$ of $\R$ to
\baav databases $\kb{\D}$ of $\kb{\R}$.
For example, consider relation schema $R(X,\ak Y,\ak Z)$ in $\R$
and mapping $g$ with $g(X) \ak =\ak X'$ and $g(Y) \ak = \ak Y'$.
Let $\kb{R}\ak \bschema{X'}{Y'}$ be a \bs in $\kb{\R}$. Then $g$
maps an instance $D$ of $R$ to a \baav instance $\kb{D}$ of $\kb{R}$
by grouping all $XY$-values in $D$ by attributes $X$, yielding
key-block pairs over $\bschema{X'}{Y'}$ for $\kb{D}$.

\vspace{0.6ex}
When $g$ maps $\R$ to $\kb{\R}$, for any database $\D$ of $\R$, we
denote by $g(\D)$ the mapping of $\D$ on $\kb{\R}$, referred to
as a {\em sibling of $\D$}.\looseness=-1

In this work, we consider \kwlog ``identity'' mappings $g$ that keep
attribute names, \ie $g(A) = A$ if $A$ is defined in $g$.



\begin{example}\label{exa-mapping}
Recall schemas $\kb{\at{T}}_{1}\bschema{(\at{date},\ak
\at{payee})}{\at{cid}}$,
$\kb{\at{C}}_{1}\bschema{\at{cid}}{\at{bid}}$, and
$\kb{\at{B}}_{1}\bschema{\at{bid}}{\at{city}}$ from
Example~\ref{exa-baav}; they are \bss. Let $\kb{\R}_{1}$ be
the set consisting of the three of them. Then $\kb{\R}_{1}$ is a
\bds and the identity mapping maps $\R_{1}$ of Example~\ref{exa-baav} to
$\kb{\R}_{1}$.
\looseness = -1

\vspace{0.36ex}
Consider \bds $\kb{\R}_{2}$ that extends $\kb{\R}_{1}$ with two
additional \bss:
$\kb{\at{B}}_{2}\bschema{\at{city}}{\at{bid}}$ and
$\kb{\at{TC}}_{1}\bschema{(\at{date}, \at{payee},
  \at{cid})}{\at{bid}}$.
Then $\kb{\R}_{2}$ is also a sibling of $\R_{1}$ as
also witnessed by the identity mapping. In particular, $\kb{\at{TC}}_{1}$
contains attributes from both $\at{T}$ and $\at{C}$ of $\R_{1}$.
\end{example}
\looseness = -1
\vspace{-0.4ex}



\begin{table}[t!]
  \caption{A summary of notations\label{tab-notation}}
  \vspace{-1ex}
  \begin{small}
    \hspace*{-0.7ex}\begin{tabular}{l|l}
      \toprule
      {\bf Notation} & {\bf Definition} \\ \toprule
      \bs (\bss) & abbrev. for \baav schema(s)\\
      \bds (\bdss) & abbrev. for \baav database schema(s), set of \bss\\
      $R$ (resp. $\R$) & a relation schema (resp. database schema)\\
      $\kb{R}$ (resp. $\kb{\R}$) & a \bs (resp. \bds)  schema \\
      $|\R|$ & the total length of relation schemas in $\R$\\
      $Q$ (resp. $\Q$) & a query (resp. set of queries, \ie workload)\\
      $|\Q|$ & \# of attributes in the $\sigma/\pi/\Join$ predicates of $\Q$\\
      $f(\Q, \kb{\R})$ & aggregate rank
      function for $\kb{\R}$ and $\Q$\\
      \bottomrule
    \end{tabular}
  \end{small}
  \vspace{-2.7ex}
\end{table}



\stitle{\baav normal form}.
Consider a set $\Q$ of parametric queries
over $\R$ and a \bds $\kb{\R}$. We say that $\kb{\R}$ is in
  {\em the normal form} (or is {\em normalized}) for $\Q$ if there exists a
mapping $g$ from $\R$ to $\kb{\R}$ such that every query $Q$ in $\Q$
can be answered over $\kb{\R}$. That is,  there exists a query plan
$\xi_{Q}$ over $\kb{\R}$ such that for any database $\D$ of $\R$,
$\xi_{Q}(\kb{\D}) = Q(\D)$, \ie the execution of 
$\xi_{Q}$ on the sibling \baav database $\kb{\D}$ of $\D$ mapped
via $g$ computes exactly the answer to $Q$ in $\D$.


Here we consider query plans $\xi$ with one bulk $\bget(T)$
operation
over $\kb{\R}$ for each relation $R$ in $\R$, \ie there
exists only one $\bget(T)$ %operation
in $\xi$ that fetches values
for each $R$. This is to ensure that $\xi$ fetches values that
are from the same tuples in an instance of $R$.

\warn{Intuitively, not all \baav schemas mapped from database schema
$\R$ can answer every query over $\R$ and the normal form
characterizes all sensible \baav schemas for $\R$ \wrt
parametric queries $\Q$.}

\begin{example}\label{exa-norm}
The \bds $\kb{\R}_{1}$ of Example~\ref{exa-mapping} is in the normal
form for $Q_{1}$ over $\R_{1}$ of Example~\ref{exa-baav}. Indeed,
the plan outlined %in
there %Example~\ref{exa-baav}
can always compute the answers
to $Q$ in any database $\D$ of $\R_{1}$ by executing it over the
sibling $\kb{\D}$ of $\D$ over \bds $\kb{\R}_{1}$.
\end{example}
\vspace{-0.4ex}

\warn{\stitle{Scan-free queries}. Consider database schema $\R$ and
parametric query $Q[P_{Q}]$ over $\R$. Let $\kb{\R}$ be a
normalized \baav schema for $\R$. Then $Q[P_{Q}]$ is {\em
  scan-free (evaluable)} if for any instantiation $Q[P_{Q} = \bar c]$ of
$Q[P_{Q}]$, there exists a scan-free query plan $\xi$ over
$\kb{\R}$ for $Q[P_{Q} = \bar c]$, \ie for any database $\D$ of
$\R$, $Q[P_{Q}=\bar c](\D) = \xi(\kb{\D})$ where \baav database
$\kb{\D}$ is the sibling of $\D$ over $\kb{\R}$.
Intuitively, we want to design normalized \baav schemas for $\R$
on which the queries are scan-free.}






\stitle{A framework}. We give a framework for speeding up
parametric queries with sibling \baav databases. Consider
workload $\Q$ of parametric queries over database schema $\R$.

\sstab (1) It computes a sibling \bds $\kb{\R}$ of $\R$ in the
normal form for $\Q$.\looseness = -1

\sstab (2) It constructs the sibling \baav database $\kb{\D}$ of
$\kb{\R}$ for the database $\D$ of $\R$, and maintains $\kb{\D}$ in
response to updates to $\D$.
\looseness = -1


\sstab (3) For any query $Q$ instantiated from parametric queries
in $\Q$, it answers $Q$ over $\kb{\D}$, benefiting from the \baav
model.



\vspace{0.8ex}
We will develop a solution to Step~(1)
in Sections~\ref{sec-rank}--\ref{sec-cover}.
For step~(3), algorithms for finding query plans
have been developed in~\cite{VLDB19}.
Below we outline a solution for step~(2).

\eetitle{Step~(2)}.
We show that
the incremental maintenance of \baav instance $\kb{D}$
of $\kb{R}$ is {\em bounded}~\cite{Reps96} when the \bs $\kb{R}\bschema{X}{Y}$
consists of attributes from the same relation of $\R$.
That is, the maintenance cost is determined by the
  size of updates, not by the cost of possibly big database $\D$.
For a set $\Delta^{+}$ of tuple insertions to $R$, updating $\kb{D}$
with $\Delta^{+}$ %$\Delta$
can be simply done by projecting $\Delta^{+}$ %$\Delta$
to $XY$
attributes and grouping them by $X$; this takes
$O(|\Delta^{+}|\log|\Delta^{+}|)$-time, where $|\Delta^{+}|$ is
the size of $\Delta^{+}$, independent of the size $|\kb{\D}|$ of
$\kb{\D}$. Similarly, deletion updates $\Delta^{-}$ can be
handled in $O(d*|\Delta^{-}|)$-time, where $d$ is the degree of
$\kb{\D}$, independent of $|\kb{\D}|$. When $\kb{R}$ contains
attributes from \eg two relations $R$ and $R'$ of $\R$ where $R$
has a foreign key for $R'$, the
%incremental
maintenance cost is dominated by the cost of
maintaining the key-foreign key join $R\Join R'$ for $\kb{R}$
which is bounded by
% \warn{$O(\cd{\Delta^{+}}|R|)$}
$O(\cd{\Delta^{+}}\cd{R'}\log(\cd{\Delta^{+}}\cd{R'}))$ (assuming
that $\Delta^{+}$ is update of tuple insertions
%is tuple insertion updates
to $R$; $\cd{R'}$ is the number of
attributes of $R'$), 
independent of $\cd{\D}$; similarly for other cases.

%\warn{Key-foreign key join should be bounded?}

Hence \baav databases can be maintained
  in a cost determined by the size of updates, no matter how
  big $\kb{D}$ grows.

\eat{%EAT
\eetitle{Remark}. The framework also supports generic \SQL
queries. Indeed, step (3) already supports generic \SQL queries
(cf.~\cite{VLDB19}). Steps (1) and (2) handle generic \SQL by
treating each query as a collection of its \SPC sub-queries.
}%EAT

\vspace{0.8ex}
Key notations of the paper are summarized in Table~\ref{tab-notation}.

